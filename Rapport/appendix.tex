\section{Appendix}
\subsection{Variables}
\begin{center}
\begin{tabular}{l l l l}
\hline 
Symbol & Variable \\ \hline
$p$ & Pitch \\
$p_c$ & Pitch setpoint \\
$\lambda$ & Travel \\
$r$ & Travel rate \\
$r_c$ & Travel rate setpoint \\
$e$ & Elevation \\
$e_c$ & Elevation setpoint \\
$V_f$ & Voltage, front motor \\
$V_b$ & Voltage, rear motor \\
$V_d$ & Voltage difference, $V_f -V_b$ \\
$V_s$ & Voltage sum, $V_f + V_b$ \\
$K_{pp}$, $K_{pd}$, $K_{ep}$, $K_{ei}$, $K_{ed}$ & Controller gains \\
$T_g$ & Moment required to keep the helicopter flying
\end{tabular}
\end{center}
\label{table:variables}



\vspace{1cm}

\subsection{Constants}
\begin{center}
\begin{tabular}{l l l l}
\hline
Symbol & Parameter & Value & Unit \\ \hline
$l_a$ & Distance from elevation axis to helicopter body & 0.64 & m \\
$l_h$ & Distance from pitch axis to motor & 0.177 & m \\
$K_f$ & Distance from elevation axis to helicopter body & 0.1983 &\si{\newton \per \meter}  \\
$J_e$ & Moment of inertia (elevation axis) & 1.0625 & \si{\kilogram \meter \squared}  \\
$J_t$ & Moment of inertia (travel axis) & 1.0625 & \si{\kilogram \meter \squared}  \\
$J_p$ & Moment of inertia (pitch axis) & 0.0406 & \si{\kilogram \meter \squared}  \\
$m_h$ & Mass of helicopter & 1.297 &\si{\kilogram}  \\
$m_w$ & Counterweight & 1.802 & \si{\kilogram} \\
$m_g$ & Effective mass of the helicopter & 0.026 & \si{\kilogram} \\
$K_p$ & Force to lift the helicopter from the ground & 0.2551 & \si{\newton}
\end{tabular}
\end{center}
\label{table:constants}

\subsection{Code for (10.2) and (10.3)}

This code is modified from the code handed out on itslearning.

\begin{lstlisting}
init04;
delta_t	  = 0.25;             % sampling time
sek_forst = 5;

q = 0.1;
% System model. x=[lambda r p p_dot]'

A1 = [1 delta_t 0 0;
      0 1 -K_2*delta_t 0;
      0 0 1 delta_t;
      0 0 -K_1*K_pp*delta_t 1-K_1*K_pd*delta_t];
  
B1 = [0; 0; 0; K_1*K_pp*delta_t];

% Number of states and inputs

mx = size(A1,2);       		  % Number of states
mu = size(B1,2);              % Number of inputs

% Initial values

x1_0 = pi;                    % Lambda
x2_0 = 0;                     % r
x3_0 = 0;                     % p
x4_0 = 0;                     % p_dot
x0   = [x1_0 x2_0 x3_0 x4_0]';% Initial values

% Time horizon and initialization

N  = 100;                     % Time horizon for states
M  = N;                       % Time horizon for inputs
z  = zeros(N*mx+M*mu,1);      % Initialize z for whole horizon
z0 = z;                       % Initial value for optimization

% Bounds

ul 	    = -30*pi/180;         % Lower bound on control -- u1
uu 	    = 30*pi/180;          % Upper bound on control -- u1

xl      = -Inf*ones(mx,1);    % Lower bound on states (no bound)
xu      = Inf*ones(mx,1);     % Upper bound on states (no bound)
xl(3)   = ul;                 % Lower bound on state x3
xu(3)   = uu;                 % Upper bound on state x3

% Generate constraints on measurements and inputs

[vlb,vub]       = genBegr2(N,M,xl,xu,ul,uu);
vlb(N*mx+M*mu)  = 0;           % Last input is zero
vub(N*mx+M*mu)  = 0;           % Last input is zero

% Generate the matrix Q and the vector c (objecitve function weights in the QP problem) 

Q1 = zeros(mx,mx);
Q1(1,1) = 1;                   % Weight on state x1
Q1(2,2) = 0;                   % Weight on state x2
Q1(3,3) = q;                   % Weight on state x3
Q1(4,4) = 0;                   % Weight on state x4
P1 = q;                        % Weight on input
Q = 2*genq2(Q1,P1,N,M,mu);     % Generate Q
c = zeros(N*mx+M*mu,1);        % Generate c

% Generate system matrixes for linear model

Aeq = gena2(A1,B1,N,mx,mu);    % Generate A
beq = zeros(1, size(Aeq,1));   % Generate b
beq(1:mx) = A1*x0; 	           % Initial value

% Solve Qp problem with linear model
tic
[z,lambda] = quadprog(Q, c, [], [], Aeq, beq, vlb, vub, z0);
t1=toc;


% Extract control inputs and states

u  = [z(N*mx+1:N*mx+M*mu);z(N*mx+M*mu)]; % Control input from solution

x1 = [x0(1);z(1:mx:N*mx)];     % State x1 from solution
x2 = [x0(2);z(2:mx:N*mx)];     % State x2 from solution
x3 = [x0(3);z(3:mx:N*mx)];     % State x3 from solution
x4 = [x0(4);z(4:mx:N*mx)];     % State x4 from solution

Antall = 5/delta_t;
Nuller = zeros(Antall,1);
Enere  = ones(Antall,1);

u   = [Nuller; u; Nuller];
x1  = [pi*Enere; x1; Nuller];
x2  = [Nuller; x2; Nuller];
x3  = [Nuller; x3; Nuller];
x4  = [Nuller; x4; Nuller];

%save trajektor1ny
t = 0:delta_t:delta_t*(length(u)-1); % real time

simin = [t' u];

% figure
                

figure(2)
subplot(511)
stairs(t,u),grid
ylabel('u')
subplot(512)
plot(t,x1,'m',t,x1,'mo'),grid
ylabel('lambda')
subplot(513)
plot(t,x2,'m',t,x2','mo'),grid
ylabel('r')
subplot(514)
plot(t,x3,'m',t,x3,'mo'),grid
ylabel('p')
subplot(515)
plot(t,x4,'m',t,x4','mo'),grid
xlabel('tid (s)'),ylabel('pdot')

x = [t' x1 x2 x3 x4];
%%

Q_k = [25 0 0 0;
       0 0.5 0 0;
       0 0 100 0;
       0 0 0 0.5];
R = 1;

[K S E] = dlqr(A1, B1, Q_k, R, 0);

\end{lstlisting}

\subsection{Code for (10.4)}

This code is modified from the code handed out on itslearning.

\begin{lstlisting}
init04;
delta_t	  = 0.25;             % sampling time
sek_forst = 5;

q = 0.1;
% System model. x=[lambda r p p_dot]'

A = [1 delta_t 0 0 0 0;
      0 1 -K_2*delta_t 0 0 0;
      0 0 1 delta_t 0 0;
      0 0 -K_1*K_pp*delta_t 1-K_1*K_pd*delta_t 0 0;
      0 0 0 0 1 delta_t;
      0 0 0 0 -delta_t*K_3*K_ep 1-delta_t*K_3*K_ed];
  
B = [0 0; 0 0; 0 0; K_1*K_pp*delta_t 0; 0 0; 0 delta_t*K_3*K_ep];

% Number of states and inputs

mx = size(A,2);               % Number of states 
mu = size(B,2);               % Number of inputs

% Initial values

x1_0 = pi;                    % Lambda
x2_0 = 0;                     % r
x3_0 = 0;                     % p
x4_0 = 0;                     % p_dot
x5_0 = 0;                     % e
x6_0 = 0;                     % e_dot
x0   = [x1_0 x2_0 x3_0 x4_0 x5_0 x6_0]'; % Initial

u1_0 = 0;
u2_0 = 0;

q_1 = 1;
q_2 = 1;
alfa = 0.2;
beta = 20;
lambda_t = 2*pi/3;

% Time horizon and initialization

N  = 40;                      % Time horizon for states
M  = N;                       % Time horizon for inputs
z  = zeros(N*mx+M*mu,1);      % Initialize z for whole horizon
z0 = z;                       % Initial value for optimization


% Bounds

ul 	    = -30*pi/180;         % Lower bound on control -- u1
uu 	    = 30*pi/180;          % Upper bound on control -- u1

xl      = -Inf*ones(mx,1);    % Lower bound on states (no bound)
xu      = Inf*ones(mx,1);     % Upper bound on states (no bound)
xl(3)   = ul;                 % Lower bound on state x3
xu(3)   = uu;                 % Upper bound on state x3

% Generate constraints on measurements and inputs

[vlb,vub]       = genBegr2(N,M,xl,xu,ul,uu);
vlb(N*mx+M*mu)  = 0;          % Last input is zero
vub(N*mx+M*mu)  = 0;          % Last input is zero

% Generate system matrixes for linear model
Aeq = gena2(A,B,N,mx,mu);     % Generate A
beq = zeros(1, size(Aeq,1));  % Generate b
beq(1:mx) = A*x0; 	       % Initial value

% Generate the matrix Q and the vector c (objecitve function weights in the QP problem) 

Q1 = zeros(mx,mx);
Q1(1,1) = 1;                  % Weight on state x1
Q1(2,2) = 0;                  % Weight on state x2
Q1(3,3) = q;                  % Weight on state x3
Q1(4,4) = 0;                  % Weight on state x4
Q1(5,5) = 0;
Q1(6,6) = 0;
P1 = zeros(mu, mu);
P1(1,1) = q_1;                % Weight on input
P1(2,2) = q_2;
Q = 2*genq2(Q1,P1,N,M,mu);    % Generate Q
c = zeros(N*mx+M*mu,1);       % Generate c

% Solve Qp problem with linear model

costf = @(z) 0.5*z'*Q*z;
%tic
%[z,lambda] = quadprog(Q, c, [], [], Aeq, beq, vlb, vub, z0);
%t1=toc;
nonlcon = @(z) alfa*exp(-beta*(z(1+mx)-lambda_t)^2)-z(5+mx);
z = fmincon(costf, z0, [], [], Aeq, beq, vlb, vub, @mycon);
% Calculate objective value

phi1 = 0.0;
PhiOut = zeros(N*mx+M*mu,1);
for i=1:N*mx+M*mu
  phi1=phi1+Q(i,i)*z(i)*z(i);
  PhiOut(i) = phi1;
end

% Extract control inputs and states

u1 = [u1_0; z(N*mx+1:mu:N*mx+M*mu)]; % Control input from solution
u2 = [u2_0; z(N*mx+2:mu:N*mx+M*mu)];

x1 = [x0(1);z(1:mx:N*mx)];      % State x1 from solution
x2 = [x0(2);z(2:mx:N*mx)];      % State x2 from solution
x3 = [x0(3);z(3:mx:N*mx)];      % State x3 from solution
x4 = [x0(4);z(4:mx:N*mx)];      % State x4 from solution
x5 = [x0(5);z(5:mx:N*mx)];
x6 = [x0(6);z(6:mx:N*mx)];

Antall = 5/delta_t;
Nuller = zeros(Antall,1);
Enere  = ones(Antall,1);

u1   = [Nuller; u1; Nuller];
u2   = [Nuller; u2; Nuller];

x1  = [x1_0*Enere; x1; Nuller];
x2  = [Nuller; x2; Nuller];
x3  = [Nuller; x3; Nuller];
x4  = [Nuller; x4; Nuller];
x5 = [Nuller; x5; Nuller];
x6 = [Nuller; x6; Nuller];

%save trajektor1ny
t = 0:delta_t:delta_t*(length(u1)-1); % real time

pc = [t' u1];
ec = [t' u2];
% figure
                

figure(2)
subplot(8,1,1)
stairs(t,u1),grid
ylabel('u1')
subplot(8,1,2)
stairs(t, u2),grid
ylabel('u2')
subplot(8,1,3)
plot(t,x1,'m',t,x1,'mo'),grid
ylabel('lambda')
subplot(8,1,4)
plot(t,x2,'m',t,x2','mo'),grid
ylabel('r')
subplot(8,1,5)
plot(t,x3,'m',t,x3,'mo'),grid
ylabel('p')
subplot(8,1,6)
plot(t,x4,'m',t,x4','mo'),grid
ylabel('pdot')
subplot(8,1,7)
plot(t,x5,'m',t,x5','mo'),grid
ylabel('e')
subplot(8,1,8)
plot(t,x6,'m',t,x6','mo'),grid
ylabel('edot')
xlabel('tid (s)')

x = [t' x1 x2 x3 x4 x5 x6];

%%
Q_k = [2 0 0 0 0 0;
       0 1 0 0 0 0;
       0 0 1 0 0 0;
       0 0 0 1.2 0 0;
       0 0 0 0 1 0;
       0 0 0 0 0 1.5];

R = [1 0;
     0 1];

[K S E] = dlqr(A, B, Q_k, R, 0);

\end{lstlisting}

Code for the non-linear constrains
\begin{lstlisting}
function [a, b] = mycon(x)

% system parameters
alpha = 0.2;
beta = 20;
lambda_t = 2*pi/3;
N = 40;

myc = @(lambda, e) alpha * exp(-beta .* (lambda - lambda_t).^2) - e;
c = @(x) myc(x(1 : 6 : 6*N), x(5 : 6 : 6*N));

a = c(x);
b = [];

end
\end{lstlisting}