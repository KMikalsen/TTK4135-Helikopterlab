
\documentclass[a4paper,12pt]{article}
\usepackage[latin1]{inputenc}
\usepackage{graphicx}
\usepackage[parfill]{parskip}
\usepackage{url}
\usepackage{amssymb,amsmath}
\usepackage{caption}
\usepackage{subcaption}
\usepackage{tabularx}
%\usepackage{floatrow}
%\usepackage{multirow}
\usepackage{cancel}
\usepackage{siunitx}
\usepackage{color}
\usepackage{arydshln}
\usepackage{epigraph}
\usepackage{booktabs}
\usepackage{fancyhdr}
\usepackage{geometry}
\usepackage{float}
\usepackage[numbered]{mcode}
\usepackage{bm}
\usepackage{hyperref}
\usepackage[round]{natbib}
\usepackage[activate={true,nocompatibility},final,tracking=true,kerning=true,spacing=true,factor=1100,stretch=10,shrink=10]{microtype}

\lstset{
    frame=single,
    breaklines=true,
    postbreak=\raisebox{0ex}[0ex][0ex]{\ensuremath{\color{green}\hookrightarrow\space}}
    breaklines=true,
    frame=topbottom
}
\hypersetup{
	colorlinks,
	citecolor=black,
	filecolor=black,
	linkcolor=black,
	urlcolor=black
}
\restylefloat{table}
\captionsetup[table]{skip=10pt}
\numberwithin{equation}{subsection}
\allowdisplaybreaks
\pagestyle{fancy}
\fancyhf[HRE,HLO]{}
    
\begin{document}
\tableofcontents

\newpage
\epigraph{\textit{"Smack my pitch up."}}{The Prodigy (1997)}
\section{Optimal control of pitch/travel with no feedback}
\subsection{State space form}
We want to write the model in equation \ref{eq:model1} in continuous time state space form with $x = \left[ \lambda\  r\  p\  \dot{p}\right]^T$ and $u=p_c$
\begin{equation}\label{eq:model1}
\dot{x}=A_cx+B_cu
\end{equation}

\newpage
\epigraph{\textit{"I've got 99 problems, but a pitch ain't one."}}{Ice-T (1993)}
\section{Optimal control of pitch/travel with LQ control}
\newpage
\section{Optimal control of pitch/travel and elevation with and without feedback}
\newpage
\section{Pastebin (remove before handing in lol)}

\subsection{Copied source for system description equations}
\begin{subequations}
\label{eq:model_al}
\begin{align}
	\ddot{e} + K_{3} K_{ed} \dot{e} + K_{3} K_{ep} e &= K_{3} K_{ep} e_{c} \label{eq:model_se_al_elev} \\
	\ddot{p} + K_{1} K_{pd} \dot{p} + K_{1} K_{pp} p &= K_{1} K_{pp} p_{c} \label{eq:model_se_al_pitch} \\
	\dot{\lambda} &= r \label{eq:model_se_al_lambda} \\
	\dot{r} &= -K_{2} p \label{eq:model_se_al_r} 
\end{align}
\end{subequations}

\end{document}
