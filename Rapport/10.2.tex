\epigraph{\textit{'Smack my pitch up.'}}{The Prodigy (1997)}

\section{Optimal control of pitch/travel without feedback}

%%%%%%%%%%%%%%%%%%%%%%%%%%%%%%%%%%%%%%%%%%%%%%%%%%%%%%%%%%%%
\subsection{State space form (10.2.1)}
%%%%%%%%%%%%%%%%%%%%%%%%%%%%%%%%%%%%%%%%%%%%%%%%%%%%%%%%%%%%
We want to write the model \eqref{eq:model1} in continuous time state space form with states and input as shown in \eqref{eq:state_and_input}.

\begin{equation}\label{eq:model1}
	\V{\dot{x}} = \M{A}_{c}\V{x} + \M{B}_{c}u
\end{equation}

\begin{subequations}\label{eq:state_and_input}
\begin{align}
	\V{x} 	&= \begin{bmatrix}\lambda & r & p & \dot{p} \end{bmatrix}\transpose \\
	u 		&= p_{c}
\end{align}
\end{subequations}

The equations of motion for the states are shown in \eqref{eq:state_equations} and the constants used are defined in \eqref{eq:K1K2}.

\begin{subequations}\label{eq:state_equations}
\begin{align}
	\dot{\lambda} 	&= r \\
	\dot{r} 		&= - K_{a} p \\
	\dot{p} 		&= \dot{p} \\
	\ddot{p} 		&= K_{1} K_{pp} (p_{c} - p) - K_{1} K_{pd} \dot{p}
\end{align}
\end{subequations}

\begin{subequations}\label{eq:K1K2}
\begin{align}
	K_{1} &= \frac{K_{f} l_{n}}{J_{p}} \\
	K_{2} &= \frac{K_{p} l_{a}}{J_{t}}
\end{align}
\end{subequations}

The above gives the result in \eqref{eq:state_space_matrices}.

\begin{equation}\label{eq:state_space_matrices}
	\V{\dot{x}} =
	\underbrace{
		\begin{bmatrix}
			0 & 1 & 0 				& 0 \\
			0 & 0 & -K_{2} 			& 0 \\
			0 & 0 & 0 				& 1 \\
			0 & 0 & -K_{1}K_{pp}	& -K_{1}K_{pd}
		\end{bmatrix}
	}_{\M{A}_{c}}
	\V{x} +
	\underbrace{
		\begin{bmatrix}
			0 \\ 0 \\ 0 \\ K_{1}K_{pp}
		\end{bmatrix}
	}_{\M{B}_{c}}
	u
\end{equation}


%%%%%%%%%%%%%%%%%%%%%%%%%%%%%%%%%%%%%%%%%%%%%%%%%%%%%%%%%%%%
\subsection{Model discussion (10.2.1)}
%%%%%%%%%%%%%%%%%%%%%%%%%%%%%%%%%%%%%%%%%%%%%%%%%%%%%%%%%%%%
The states are travel, travel rate, pitch, and pitch rate. The controller output is the pitch setpoint to be used by the already implemented controller which in turn calculates voltage inputs for the plant. Thus, we are modelling not the helicopter alone, but a system that consists of the helicopter along with the given controller. This corresponds with Figure 7 (proper ref here?) in the exercise text (add ref).


%%%%%%%%%%%%%%%%%%%%%%%%%%%%%%%%%%%%%%%%%%%%%%%%%%%%%%%%%%%%
\subsection{Discretisation (10.2.2)}
%%%%%%%%%%%%%%%%%%%%%%%%%%%%%%%%%%%%%%%%%%%%%%%%%%%%%%%%%%%%
We discretise the system by the Forward Euler Method. The general definition of the method and how it relates to our system is described by equations \eqref{eq:forward_euler} and \eqref{eq:euler_func}, respectively.

\begin{equation}\label{eq:forward_euler}
	y_{k+1} = y_{k} + h f(x_{k}, y_{k})
\end{equation}

\begin{equation}\label{eq:euler_func}
	f = \left( \M{A}_{c} \V{x}_{k} + \M{B}_c u_{k} \right)
\end{equation}

Using \eqref{eq:forward_euler} and \eqref{eq:euler_func}, we can find the matrices of the discretised system, as shown in \eqref{eq:disc} and \eqref{eq:disc_matrices}.

\begin{subequations}\label{eq:disc}
\begin{align}
	\V{x}_{k+1} &= \V{x}_{k} + \left( \M{A}_{c} \V{x}_{k} + \M{B}_c u_{k} \right) h \\
				&= \V{x}_{k} + h\M{A}_{c}\V{x}_{k} + h\M{B}_{c}u_{k} \\
				&= \left( \Mc{I} + h\M{A}_{c} \right) \V{x}_{k} + h\M{B}_{c}u_{k} \\
				&= \M{A}\V{x}_{k} + \M{B}u_{k}
\end{align}
\end{subequations}

\begin{subequations}\label{eq:disc_matrices}
\begin{equation}
	\M{A} = \Mc{I} + h\M{A}_{c} =
	\begin{bmatrix}
		1 & h & 0 & 0 \\
		0 & 1 & -K_{2}h & 0 \\
		0 & 0 & 1 & h \\
		0 & 0 & -K_{1}K_{pp}h	& 1-K_{1}K_{pd}h
	\end{bmatrix}
\end{equation}
\begin{equation}
	\M{B} = h\M{B}_c =
	\begin{bmatrix} 0 \\ 0 \\ 0 \\ K_{1}K_{pp}h \end{bmatrix}
\end{equation}
\end{subequations}

%%%%%%%%%%%%%%%%%%%%%%%%%%%%%%%%%%%%%%%%%%%%%%%%%%%%%%%%%%%%
\subsection{Cost function discussion}
%%%%%%%%%%%%%%%%%%%%%%%%%%%%%%%%%%%%%%%%%%%%%%%%%%%%%%%%%%%%
- cost function has a focus on the error between $\lambda_i$ and $\lambda_f$
- is a quadratic function, least-square 

%%%%%%%%%%%%%%%%%%%%%%%%%%%%%%%%%%%%%%%%%%%%%%%%%%%%%%%%%%%%
\subsection{unwatend effect dicsussion}
%%%%%%%%%%%%%%%%%%%%%%%%%%%%%%%%%%%%%%%%%%%%%%%%%%%%%%%%%%%%
- when $\lambda = \lambda_f$, you will minimize the pitch, making it 0. This can make the helicopter oscilate around $\lambda = \lambda_f$, as the pitch will be set to 0 on that particular point

%%%%%%%%%%%%%%%%%%%%%%%%%%%%%%%%%%%%%%%%%%%%%%%%%%%%%%%%%%%%
\subsubsection{4 does the helicopter end in desired and what causes deviation}
%%%%%%%%%%%%%%%%%%%%%%%%%%%%%%%%%%%%%%%%%%%%%%%%%%%%%%%%%%%%
- No, because the optimization calculation doesn't take in account the brake time to stop in the desired point, and thus it doesn't stop in time.
